\documentclass[11pt]{article}

    \usepackage[breakable]{tcolorbox}
    \usepackage{parskip} % Stop auto-indenting (to mimic markdown behaviour)
    

    % Basic figure setup, for now with no caption control since it's done
    % automatically by Pandoc (which extracts ![](path) syntax from Markdown).
    \usepackage{graphicx}
    % Maintain compatibility with old templates. Remove in nbconvert 6.0
    \let\Oldincludegraphics\includegraphics
    % Ensure that by default, figures have no caption (until we provide a
    % proper Figure object with a Caption API and a way to capture that
    % in the conversion process - todo).
    \usepackage{caption}
    \DeclareCaptionFormat{nocaption}{}
    \captionsetup{format=nocaption,aboveskip=0pt,belowskip=0pt}

    \usepackage{float}
    \floatplacement{figure}{H} % forces figures to be placed at the correct location
    \usepackage{xcolor} % Allow colors to be defined
    \usepackage{enumerate} % Needed for markdown enumerations to work
    \usepackage{geometry} % Used to adjust the document margins
    \usepackage{amsmath} % Equations
    \usepackage{amssymb} % Equations
    \usepackage{textcomp} % defines textquotesingle
    % Hack from http://tex.stackexchange.com/a/47451/13684:
    \AtBeginDocument{%
        \def\PYZsq{\textquotesingle}% Upright quotes in Pygmentized code
    }
    \usepackage{upquote} % Upright quotes for verbatim code
    \usepackage{eurosym} % defines \euro

    \usepackage{iftex}
    \ifPDFTeX
        \usepackage[T1]{fontenc}
        \IfFileExists{alphabeta.sty}{
              \usepackage{alphabeta}
          }{
              \usepackage[mathletters]{ucs}
              \usepackage[utf8x]{inputenc}
          }
    \else
        \usepackage{fontspec}
        \usepackage{unicode-math}
    \fi

    \usepackage{fancyvrb} % verbatim replacement that allows latex
    \usepackage{grffile} % extends the file name processing of package graphics 
                         % to support a larger range
    \makeatletter % fix for old versions of grffile with XeLaTeX
    \@ifpackagelater{grffile}{2019/11/01}
    {
      % Do nothing on new versions
    }
    {
      \def\Gread@@xetex#1{%
        \IfFileExists{"\Gin@base".bb}%
        {\Gread@eps{\Gin@base.bb}}%
        {\Gread@@xetex@aux#1}%
      }
    }
    \makeatother
    \usepackage[Export]{adjustbox} % Used to constrain images to a maximum size
    \adjustboxset{max size={0.9\linewidth}{0.9\paperheight}}

    % The hyperref package gives us a pdf with properly built
    % internal navigation ('pdf bookmarks' for the table of contents,
    % internal cross-reference links, web links for URLs, etc.)
    \usepackage{hyperref}
    % The default LaTeX title has an obnoxious amount of whitespace. By default,
    % titling removes some of it. It also provides customization options.
    \usepackage{titling}
    \usepackage{longtable} % longtable support required by pandoc >1.10
    \usepackage{booktabs}  % table support for pandoc > 1.12.2
    \usepackage{array}     % table support for pandoc >= 2.11.3
    \usepackage{calc}      % table minipage width calculation for pandoc >= 2.11.1
    \usepackage[inline]{enumitem} % IRkernel/repr support (it uses the enumerate* environment)
    \usepackage[normalem]{ulem} % ulem is needed to support strikethroughs (\sout)
                                % normalem makes italics be italics, not underlines
    \usepackage{mathrsfs}
    

    
    % Colors for the hyperref package
    \definecolor{urlcolor}{rgb}{0,.145,.698}
    \definecolor{linkcolor}{rgb}{.71,0.21,0.01}
    \definecolor{citecolor}{rgb}{.12,.54,.11}

    % ANSI colors
    \definecolor{ansi-black}{HTML}{3E424D}
    \definecolor{ansi-black-intense}{HTML}{282C36}
    \definecolor{ansi-red}{HTML}{E75C58}
    \definecolor{ansi-red-intense}{HTML}{B22B31}
    \definecolor{ansi-green}{HTML}{00A250}
    \definecolor{ansi-green-intense}{HTML}{007427}
    \definecolor{ansi-yellow}{HTML}{DDB62B}
    \definecolor{ansi-yellow-intense}{HTML}{B27D12}
    \definecolor{ansi-blue}{HTML}{208FFB}
    \definecolor{ansi-blue-intense}{HTML}{0065CA}
    \definecolor{ansi-magenta}{HTML}{D160C4}
    \definecolor{ansi-magenta-intense}{HTML}{A03196}
    \definecolor{ansi-cyan}{HTML}{60C6C8}
    \definecolor{ansi-cyan-intense}{HTML}{258F8F}
    \definecolor{ansi-white}{HTML}{C5C1B4}
    \definecolor{ansi-white-intense}{HTML}{A1A6B2}
    \definecolor{ansi-default-inverse-fg}{HTML}{FFFFFF}
    \definecolor{ansi-default-inverse-bg}{HTML}{000000}

    % common color for the border for error outputs.
    \definecolor{outerrorbackground}{HTML}{FFDFDF}

    % commands and environments needed by pandoc snippets
    % extracted from the output of `pandoc -s`
    \providecommand{\tightlist}{%
      \setlength{\itemsep}{0pt}\setlength{\parskip}{0pt}}
    \DefineVerbatimEnvironment{Highlighting}{Verbatim}{commandchars=\\\{\}}
    % Add ',fontsize=\small' for more characters per line
    \newenvironment{Shaded}{}{}
    \newcommand{\KeywordTok}[1]{\textcolor[rgb]{0.00,0.44,0.13}{\textbf{{#1}}}}
    \newcommand{\DataTypeTok}[1]{\textcolor[rgb]{0.56,0.13,0.00}{{#1}}}
    \newcommand{\DecValTok}[1]{\textcolor[rgb]{0.25,0.63,0.44}{{#1}}}
    \newcommand{\BaseNTok}[1]{\textcolor[rgb]{0.25,0.63,0.44}{{#1}}}
    \newcommand{\FloatTok}[1]{\textcolor[rgb]{0.25,0.63,0.44}{{#1}}}
    \newcommand{\CharTok}[1]{\textcolor[rgb]{0.25,0.44,0.63}{{#1}}}
    \newcommand{\StringTok}[1]{\textcolor[rgb]{0.25,0.44,0.63}{{#1}}}
    \newcommand{\CommentTok}[1]{\textcolor[rgb]{0.38,0.63,0.69}{\textit{{#1}}}}
    \newcommand{\OtherTok}[1]{\textcolor[rgb]{0.00,0.44,0.13}{{#1}}}
    \newcommand{\AlertTok}[1]{\textcolor[rgb]{1.00,0.00,0.00}{\textbf{{#1}}}}
    \newcommand{\FunctionTok}[1]{\textcolor[rgb]{0.02,0.16,0.49}{{#1}}}
    \newcommand{\RegionMarkerTok}[1]{{#1}}
    \newcommand{\ErrorTok}[1]{\textcolor[rgb]{1.00,0.00,0.00}{\textbf{{#1}}}}
    \newcommand{\NormalTok}[1]{{#1}}
    
    % Additional commands for more recent versions of Pandoc
    \newcommand{\ConstantTok}[1]{\textcolor[rgb]{0.53,0.00,0.00}{{#1}}}
    \newcommand{\SpecialCharTok}[1]{\textcolor[rgb]{0.25,0.44,0.63}{{#1}}}
    \newcommand{\VerbatimStringTok}[1]{\textcolor[rgb]{0.25,0.44,0.63}{{#1}}}
    \newcommand{\SpecialStringTok}[1]{\textcolor[rgb]{0.73,0.40,0.53}{{#1}}}
    \newcommand{\ImportTok}[1]{{#1}}
    \newcommand{\DocumentationTok}[1]{\textcolor[rgb]{0.73,0.13,0.13}{\textit{{#1}}}}
    \newcommand{\AnnotationTok}[1]{\textcolor[rgb]{0.38,0.63,0.69}{\textbf{\textit{{#1}}}}}
    \newcommand{\CommentVarTok}[1]{\textcolor[rgb]{0.38,0.63,0.69}{\textbf{\textit{{#1}}}}}
    \newcommand{\VariableTok}[1]{\textcolor[rgb]{0.10,0.09,0.49}{{#1}}}
    \newcommand{\ControlFlowTok}[1]{\textcolor[rgb]{0.00,0.44,0.13}{\textbf{{#1}}}}
    \newcommand{\OperatorTok}[1]{\textcolor[rgb]{0.40,0.40,0.40}{{#1}}}
    \newcommand{\BuiltInTok}[1]{{#1}}
    \newcommand{\ExtensionTok}[1]{{#1}}
    \newcommand{\PreprocessorTok}[1]{\textcolor[rgb]{0.74,0.48,0.00}{{#1}}}
    \newcommand{\AttributeTok}[1]{\textcolor[rgb]{0.49,0.56,0.16}{{#1}}}
    \newcommand{\InformationTok}[1]{\textcolor[rgb]{0.38,0.63,0.69}{\textbf{\textit{{#1}}}}}
    \newcommand{\WarningTok}[1]{\textcolor[rgb]{0.38,0.63,0.69}{\textbf{\textit{{#1}}}}}
    
    
    % Define a nice break command that doesn't care if a line doesn't already
    % exist.
    \def\br{\hspace*{\fill} \\* }
    % Math Jax compatibility definitions
    \def\gt{>}
    \def\lt{<}
    \let\Oldtex\TeX
    \let\Oldlatex\LaTeX
    \renewcommand{\TeX}{\textrm{\Oldtex}}
    \renewcommand{\LaTeX}{\textrm{\Oldlatex}}
    % Document parameters
    % Document title
    \title{Data\_Types\_and\_Operators}
    
    
    
    
    
% Pygments definitions
\makeatletter
\def\PY@reset{\let\PY@it=\relax \let\PY@bf=\relax%
    \let\PY@ul=\relax \let\PY@tc=\relax%
    \let\PY@bc=\relax \let\PY@ff=\relax}
\def\PY@tok#1{\csname PY@tok@#1\endcsname}
\def\PY@toks#1+{\ifx\relax#1\empty\else%
    \PY@tok{#1}\expandafter\PY@toks\fi}
\def\PY@do#1{\PY@bc{\PY@tc{\PY@ul{%
    \PY@it{\PY@bf{\PY@ff{#1}}}}}}}
\def\PY#1#2{\PY@reset\PY@toks#1+\relax+\PY@do{#2}}

\@namedef{PY@tok@w}{\def\PY@tc##1{\textcolor[rgb]{0.73,0.73,0.73}{##1}}}
\@namedef{PY@tok@c}{\let\PY@it=\textit\def\PY@tc##1{\textcolor[rgb]{0.24,0.48,0.48}{##1}}}
\@namedef{PY@tok@cp}{\def\PY@tc##1{\textcolor[rgb]{0.61,0.40,0.00}{##1}}}
\@namedef{PY@tok@k}{\let\PY@bf=\textbf\def\PY@tc##1{\textcolor[rgb]{0.00,0.50,0.00}{##1}}}
\@namedef{PY@tok@kp}{\def\PY@tc##1{\textcolor[rgb]{0.00,0.50,0.00}{##1}}}
\@namedef{PY@tok@kt}{\def\PY@tc##1{\textcolor[rgb]{0.69,0.00,0.25}{##1}}}
\@namedef{PY@tok@o}{\def\PY@tc##1{\textcolor[rgb]{0.40,0.40,0.40}{##1}}}
\@namedef{PY@tok@ow}{\let\PY@bf=\textbf\def\PY@tc##1{\textcolor[rgb]{0.67,0.13,1.00}{##1}}}
\@namedef{PY@tok@nb}{\def\PY@tc##1{\textcolor[rgb]{0.00,0.50,0.00}{##1}}}
\@namedef{PY@tok@nf}{\def\PY@tc##1{\textcolor[rgb]{0.00,0.00,1.00}{##1}}}
\@namedef{PY@tok@nc}{\let\PY@bf=\textbf\def\PY@tc##1{\textcolor[rgb]{0.00,0.00,1.00}{##1}}}
\@namedef{PY@tok@nn}{\let\PY@bf=\textbf\def\PY@tc##1{\textcolor[rgb]{0.00,0.00,1.00}{##1}}}
\@namedef{PY@tok@ne}{\let\PY@bf=\textbf\def\PY@tc##1{\textcolor[rgb]{0.80,0.25,0.22}{##1}}}
\@namedef{PY@tok@nv}{\def\PY@tc##1{\textcolor[rgb]{0.10,0.09,0.49}{##1}}}
\@namedef{PY@tok@no}{\def\PY@tc##1{\textcolor[rgb]{0.53,0.00,0.00}{##1}}}
\@namedef{PY@tok@nl}{\def\PY@tc##1{\textcolor[rgb]{0.46,0.46,0.00}{##1}}}
\@namedef{PY@tok@ni}{\let\PY@bf=\textbf\def\PY@tc##1{\textcolor[rgb]{0.44,0.44,0.44}{##1}}}
\@namedef{PY@tok@na}{\def\PY@tc##1{\textcolor[rgb]{0.41,0.47,0.13}{##1}}}
\@namedef{PY@tok@nt}{\let\PY@bf=\textbf\def\PY@tc##1{\textcolor[rgb]{0.00,0.50,0.00}{##1}}}
\@namedef{PY@tok@nd}{\def\PY@tc##1{\textcolor[rgb]{0.67,0.13,1.00}{##1}}}
\@namedef{PY@tok@s}{\def\PY@tc##1{\textcolor[rgb]{0.73,0.13,0.13}{##1}}}
\@namedef{PY@tok@sd}{\let\PY@it=\textit\def\PY@tc##1{\textcolor[rgb]{0.73,0.13,0.13}{##1}}}
\@namedef{PY@tok@si}{\let\PY@bf=\textbf\def\PY@tc##1{\textcolor[rgb]{0.64,0.35,0.47}{##1}}}
\@namedef{PY@tok@se}{\let\PY@bf=\textbf\def\PY@tc##1{\textcolor[rgb]{0.67,0.36,0.12}{##1}}}
\@namedef{PY@tok@sr}{\def\PY@tc##1{\textcolor[rgb]{0.64,0.35,0.47}{##1}}}
\@namedef{PY@tok@ss}{\def\PY@tc##1{\textcolor[rgb]{0.10,0.09,0.49}{##1}}}
\@namedef{PY@tok@sx}{\def\PY@tc##1{\textcolor[rgb]{0.00,0.50,0.00}{##1}}}
\@namedef{PY@tok@m}{\def\PY@tc##1{\textcolor[rgb]{0.40,0.40,0.40}{##1}}}
\@namedef{PY@tok@gh}{\let\PY@bf=\textbf\def\PY@tc##1{\textcolor[rgb]{0.00,0.00,0.50}{##1}}}
\@namedef{PY@tok@gu}{\let\PY@bf=\textbf\def\PY@tc##1{\textcolor[rgb]{0.50,0.00,0.50}{##1}}}
\@namedef{PY@tok@gd}{\def\PY@tc##1{\textcolor[rgb]{0.63,0.00,0.00}{##1}}}
\@namedef{PY@tok@gi}{\def\PY@tc##1{\textcolor[rgb]{0.00,0.52,0.00}{##1}}}
\@namedef{PY@tok@gr}{\def\PY@tc##1{\textcolor[rgb]{0.89,0.00,0.00}{##1}}}
\@namedef{PY@tok@ge}{\let\PY@it=\textit}
\@namedef{PY@tok@gs}{\let\PY@bf=\textbf}
\@namedef{PY@tok@gp}{\let\PY@bf=\textbf\def\PY@tc##1{\textcolor[rgb]{0.00,0.00,0.50}{##1}}}
\@namedef{PY@tok@go}{\def\PY@tc##1{\textcolor[rgb]{0.44,0.44,0.44}{##1}}}
\@namedef{PY@tok@gt}{\def\PY@tc##1{\textcolor[rgb]{0.00,0.27,0.87}{##1}}}
\@namedef{PY@tok@err}{\def\PY@bc##1{{\setlength{\fboxsep}{\string -\fboxrule}\fcolorbox[rgb]{1.00,0.00,0.00}{1,1,1}{\strut ##1}}}}
\@namedef{PY@tok@kc}{\let\PY@bf=\textbf\def\PY@tc##1{\textcolor[rgb]{0.00,0.50,0.00}{##1}}}
\@namedef{PY@tok@kd}{\let\PY@bf=\textbf\def\PY@tc##1{\textcolor[rgb]{0.00,0.50,0.00}{##1}}}
\@namedef{PY@tok@kn}{\let\PY@bf=\textbf\def\PY@tc##1{\textcolor[rgb]{0.00,0.50,0.00}{##1}}}
\@namedef{PY@tok@kr}{\let\PY@bf=\textbf\def\PY@tc##1{\textcolor[rgb]{0.00,0.50,0.00}{##1}}}
\@namedef{PY@tok@bp}{\def\PY@tc##1{\textcolor[rgb]{0.00,0.50,0.00}{##1}}}
\@namedef{PY@tok@fm}{\def\PY@tc##1{\textcolor[rgb]{0.00,0.00,1.00}{##1}}}
\@namedef{PY@tok@vc}{\def\PY@tc##1{\textcolor[rgb]{0.10,0.09,0.49}{##1}}}
\@namedef{PY@tok@vg}{\def\PY@tc##1{\textcolor[rgb]{0.10,0.09,0.49}{##1}}}
\@namedef{PY@tok@vi}{\def\PY@tc##1{\textcolor[rgb]{0.10,0.09,0.49}{##1}}}
\@namedef{PY@tok@vm}{\def\PY@tc##1{\textcolor[rgb]{0.10,0.09,0.49}{##1}}}
\@namedef{PY@tok@sa}{\def\PY@tc##1{\textcolor[rgb]{0.73,0.13,0.13}{##1}}}
\@namedef{PY@tok@sb}{\def\PY@tc##1{\textcolor[rgb]{0.73,0.13,0.13}{##1}}}
\@namedef{PY@tok@sc}{\def\PY@tc##1{\textcolor[rgb]{0.73,0.13,0.13}{##1}}}
\@namedef{PY@tok@dl}{\def\PY@tc##1{\textcolor[rgb]{0.73,0.13,0.13}{##1}}}
\@namedef{PY@tok@s2}{\def\PY@tc##1{\textcolor[rgb]{0.73,0.13,0.13}{##1}}}
\@namedef{PY@tok@sh}{\def\PY@tc##1{\textcolor[rgb]{0.73,0.13,0.13}{##1}}}
\@namedef{PY@tok@s1}{\def\PY@tc##1{\textcolor[rgb]{0.73,0.13,0.13}{##1}}}
\@namedef{PY@tok@mb}{\def\PY@tc##1{\textcolor[rgb]{0.40,0.40,0.40}{##1}}}
\@namedef{PY@tok@mf}{\def\PY@tc##1{\textcolor[rgb]{0.40,0.40,0.40}{##1}}}
\@namedef{PY@tok@mh}{\def\PY@tc##1{\textcolor[rgb]{0.40,0.40,0.40}{##1}}}
\@namedef{PY@tok@mi}{\def\PY@tc##1{\textcolor[rgb]{0.40,0.40,0.40}{##1}}}
\@namedef{PY@tok@il}{\def\PY@tc##1{\textcolor[rgb]{0.40,0.40,0.40}{##1}}}
\@namedef{PY@tok@mo}{\def\PY@tc##1{\textcolor[rgb]{0.40,0.40,0.40}{##1}}}
\@namedef{PY@tok@ch}{\let\PY@it=\textit\def\PY@tc##1{\textcolor[rgb]{0.24,0.48,0.48}{##1}}}
\@namedef{PY@tok@cm}{\let\PY@it=\textit\def\PY@tc##1{\textcolor[rgb]{0.24,0.48,0.48}{##1}}}
\@namedef{PY@tok@cpf}{\let\PY@it=\textit\def\PY@tc##1{\textcolor[rgb]{0.24,0.48,0.48}{##1}}}
\@namedef{PY@tok@c1}{\let\PY@it=\textit\def\PY@tc##1{\textcolor[rgb]{0.24,0.48,0.48}{##1}}}
\@namedef{PY@tok@cs}{\let\PY@it=\textit\def\PY@tc##1{\textcolor[rgb]{0.24,0.48,0.48}{##1}}}

\def\PYZbs{\char`\\}
\def\PYZus{\char`\_}
\def\PYZob{\char`\{}
\def\PYZcb{\char`\}}
\def\PYZca{\char`\^}
\def\PYZam{\char`\&}
\def\PYZlt{\char`\<}
\def\PYZgt{\char`\>}
\def\PYZsh{\char`\#}
\def\PYZpc{\char`\%}
\def\PYZdl{\char`\$}
\def\PYZhy{\char`\-}
\def\PYZsq{\char`\'}
\def\PYZdq{\char`\"}
\def\PYZti{\char`\~}
% for compatibility with earlier versions
\def\PYZat{@}
\def\PYZlb{[}
\def\PYZrb{]}
\makeatother


    % For linebreaks inside Verbatim environment from package fancyvrb. 
    \makeatletter
        \newbox\Wrappedcontinuationbox 
        \newbox\Wrappedvisiblespacebox 
        \newcommand*\Wrappedvisiblespace {\textcolor{red}{\textvisiblespace}} 
        \newcommand*\Wrappedcontinuationsymbol {\textcolor{red}{\llap{\tiny$\m@th\hookrightarrow$}}} 
        \newcommand*\Wrappedcontinuationindent {3ex } 
        \newcommand*\Wrappedafterbreak {\kern\Wrappedcontinuationindent\copy\Wrappedcontinuationbox} 
        % Take advantage of the already applied Pygments mark-up to insert 
        % potential linebreaks for TeX processing. 
        %        {, <, #, %, $, ' and ": go to next line. 
        %        _, }, ^, &, >, - and ~: stay at end of broken line. 
        % Use of \textquotesingle for straight quote. 
        \newcommand*\Wrappedbreaksatspecials {% 
            \def\PYGZus{\discretionary{\char`\_}{\Wrappedafterbreak}{\char`\_}}% 
            \def\PYGZob{\discretionary{}{\Wrappedafterbreak\char`\{}{\char`\{}}% 
            \def\PYGZcb{\discretionary{\char`\}}{\Wrappedafterbreak}{\char`\}}}% 
            \def\PYGZca{\discretionary{\char`\^}{\Wrappedafterbreak}{\char`\^}}% 
            \def\PYGZam{\discretionary{\char`\&}{\Wrappedafterbreak}{\char`\&}}% 
            \def\PYGZlt{\discretionary{}{\Wrappedafterbreak\char`\<}{\char`\<}}% 
            \def\PYGZgt{\discretionary{\char`\>}{\Wrappedafterbreak}{\char`\>}}% 
            \def\PYGZsh{\discretionary{}{\Wrappedafterbreak\char`\#}{\char`\#}}% 
            \def\PYGZpc{\discretionary{}{\Wrappedafterbreak\char`\%}{\char`\%}}% 
            \def\PYGZdl{\discretionary{}{\Wrappedafterbreak\char`\$}{\char`\$}}% 
            \def\PYGZhy{\discretionary{\char`\-}{\Wrappedafterbreak}{\char`\-}}% 
            \def\PYGZsq{\discretionary{}{\Wrappedafterbreak\textquotesingle}{\textquotesingle}}% 
            \def\PYGZdq{\discretionary{}{\Wrappedafterbreak\char`\"}{\char`\"}}% 
            \def\PYGZti{\discretionary{\char`\~}{\Wrappedafterbreak}{\char`\~}}% 
        } 
        % Some characters . , ; ? ! / are not pygmentized. 
        % This macro makes them "active" and they will insert potential linebreaks 
        \newcommand*\Wrappedbreaksatpunct {% 
            \lccode`\~`\.\lowercase{\def~}{\discretionary{\hbox{\char`\.}}{\Wrappedafterbreak}{\hbox{\char`\.}}}% 
            \lccode`\~`\,\lowercase{\def~}{\discretionary{\hbox{\char`\,}}{\Wrappedafterbreak}{\hbox{\char`\,}}}% 
            \lccode`\~`\;\lowercase{\def~}{\discretionary{\hbox{\char`\;}}{\Wrappedafterbreak}{\hbox{\char`\;}}}% 
            \lccode`\~`\:\lowercase{\def~}{\discretionary{\hbox{\char`\:}}{\Wrappedafterbreak}{\hbox{\char`\:}}}% 
            \lccode`\~`\?\lowercase{\def~}{\discretionary{\hbox{\char`\?}}{\Wrappedafterbreak}{\hbox{\char`\?}}}% 
            \lccode`\~`\!\lowercase{\def~}{\discretionary{\hbox{\char`\!}}{\Wrappedafterbreak}{\hbox{\char`\!}}}% 
            \lccode`\~`\/\lowercase{\def~}{\discretionary{\hbox{\char`\/}}{\Wrappedafterbreak}{\hbox{\char`\/}}}% 
            \catcode`\.\active
            \catcode`\,\active 
            \catcode`\;\active
            \catcode`\:\active
            \catcode`\?\active
            \catcode`\!\active
            \catcode`\/\active 
            \lccode`\~`\~ 	
        }
    \makeatother

    \let\OriginalVerbatim=\Verbatim
    \makeatletter
    \renewcommand{\Verbatim}[1][1]{%
        %\parskip\z@skip
        \sbox\Wrappedcontinuationbox {\Wrappedcontinuationsymbol}%
        \sbox\Wrappedvisiblespacebox {\FV@SetupFont\Wrappedvisiblespace}%
        \def\FancyVerbFormatLine ##1{\hsize\linewidth
            \vtop{\raggedright\hyphenpenalty\z@\exhyphenpenalty\z@
                \doublehyphendemerits\z@\finalhyphendemerits\z@
                \strut ##1\strut}%
        }%
        % If the linebreak is at a space, the latter will be displayed as visible
        % space at end of first line, and a continuation symbol starts next line.
        % Stretch/shrink are however usually zero for typewriter font.
        \def\FV@Space {%
            \nobreak\hskip\z@ plus\fontdimen3\font minus\fontdimen4\font
            \discretionary{\copy\Wrappedvisiblespacebox}{\Wrappedafterbreak}
            {\kern\fontdimen2\font}%
        }%
        
        % Allow breaks at special characters using \PYG... macros.
        \Wrappedbreaksatspecials
        % Breaks at punctuation characters . , ; ? ! and / need catcode=\active 	
        \OriginalVerbatim[#1,codes*=\Wrappedbreaksatpunct]%
    }
    \makeatother

    % Exact colors from NB
    \definecolor{incolor}{HTML}{303F9F}
    \definecolor{outcolor}{HTML}{D84315}
    \definecolor{cellborder}{HTML}{CFCFCF}
    \definecolor{cellbackground}{HTML}{F7F7F7}
    
    % prompt
    \makeatletter
    \newcommand{\boxspacing}{\kern\kvtcb@left@rule\kern\kvtcb@boxsep}
    \makeatother
    \newcommand{\prompt}[4]{
        {\ttfamily\llap{{\color{#2}[#3]:\hspace{3pt}#4}}\vspace{-\baselineskip}}
    }
    

    
    % Prevent overflowing lines due to hard-to-break entities
    \sloppy 
    % Setup hyperref package
    \hypersetup{
      breaklinks=true,  % so long urls are correctly broken across lines
      colorlinks=true,
      urlcolor=urlcolor,
      linkcolor=linkcolor,
      citecolor=citecolor,
      }
    % Slightly bigger margins than the latex defaults
    
    \geometry{verbose,tmargin=1in,bmargin=1in,lmargin=1in,rmargin=1in}
    
    

\begin{document}
    
    \maketitle
    
    

    
    \hypertarget{welcome-to-introduction-to-python}{%
\section{Welcome to Introduction to
Python!}\label{welcome-to-introduction-to-python}}

    In this course, we use Python version 3 (or simply Python 3). If you'd
like more details on previous versions of Python and how version 3
differs from previous versions, check out the
\href{https://en.wikipedia.org/wiki/History_of_Python}{History of
Python} on Wikipedia. If you're new to Python or programming in general,
this article will make more sense after you've completed a lesson or
two, so you may want to hold off for now. All you need to know now is
that your solution code for the programming exercises. in this course
will be graded based on Python 3 code.

    \hypertarget{programming-in-python}{%
\subsection{Programming in Python}\label{programming-in-python}}

As you learn Python throughout this course, there are a few things you
should keep in mind. 1. Python is case sensitive. 2. Spacing is
important. 3. Use error messages to help you learn.

Let's get started!

    \hypertarget{summary}{%
\subsection{Summary}\label{summary}}

    \hypertarget{lesson-summary}{%
\subsubsection{Lesson Summary}\label{lesson-summary}}

You learned a ton in this lesson! To summarize, here's a recap of the
data types and operators we covered.

\hypertarget{data-types}{%
\subsubsection{Data Types}\label{data-types}}

We covered four important data types that you'll use all the time in
programming:

\begin{enumerate}
\def\labelenumi{\arabic{enumi}.}
\tightlist
\item
  Data Type
\item
  Constructor
\item
  Example:int \textbar{} int() float \textbar{} float() \textbar{}
  string \textbar{} '\,' or ``\,'' or str() \textbar{} ``this is a
  string'' \textbar{} \textbar{} bool \textbar{} bool() \textbar{} True
  or False \textbar{}
\end{enumerate}

    \hypertarget{operators}{%
\subsubsection{Operators}\label{operators}}

We also covered four useful sets of operators:

    

    

    \hypertarget{data-types-and-operators}{%
\subsection{Data Types and Operators}\label{data-types-and-operators}}

Welcome to this lesson on Data Types and Operators! You'll learn about:

\begin{enumerate}
\def\labelenumi{\arabic{enumi}.}
\tightlist
\item
  Data Types: Integers, Floats, Booleans, Strings
\item
  Operators: Arithmetic, Assignment, Comparison, Logical
\item
  Built-In Functions, Type Conversion
\item
  Whitespace and Style Guidelines
\end{enumerate}

    \hypertarget{arithmetic-operators}{%
\subsubsection{Arithmetic Operators}\label{arithmetic-operators}}

Arithmetic operators: 1. {[}+{]} Addition. 2. {[}-{]} Subtraction. 3.
{[}*{]} Multiplication. 4. {[}/{]} Division. 5. {[}\%{]} Mod (the
remainder after dividing). 6. {[}**{]} Exponentiation (note that \^{}
does not do this operation, as you might have seen in other languages).
7. {[}//{]} Divides and rounds down to the nearest integer.

The usual order of mathematical operations holds in Python, which you
can review in this Math Forum
\href{https://www.nctm.org/classroomresources/}{page} if needed.

Bitwise operators are special operators in Python that you can learn
more about \href{https://wiki.python.org/moin/BitwiseOperators}{here} if
you'd like.

\hypertarget{examples}{%
\paragraph{Examples:}\label{examples}}

\begin{enumerate}
\def\labelenumi{\arabic{enumi}.}
\tightlist
\item
  print(3 + 5) \# 8
\item
  print(1 + 2 + 3 * 3) \# 12
\item
  print(3 ** 2) \# 9
\item
  print(9 \% 2) \# 1
\end{enumerate}

    \hypertarget{quiz-arithmetic-operators}{%
\subsubsection{Quiz: Arithmetic
Operators}\label{quiz-arithmetic-operators}}

Quiz: Average Electricity Bill It's time to try a calculation in Python!

My electricity bills for the last three months have been \$23, \$32 and
\$64. What is the average monthly electricity bill over the three month
period? Write an expression to calculate the mean, and use print() to
view the result.

    \begin{tcolorbox}[breakable, size=fbox, boxrule=1pt, pad at break*=1mm,colback=cellbackground, colframe=cellborder]
\prompt{In}{incolor}{1}{\boxspacing}
\begin{Verbatim}[commandchars=\\\{\}]
\PY{c+c1}{\PYZsh{} Write an expression that calculates the average of 23, 32 and 64}
\PY{n}{summation} \PY{o}{=} \PY{l+m+mi}{23} \PY{o}{+} \PY{l+m+mi}{32} \PY{o}{+} \PY{l+m+mi}{64}
\PY{n}{num\PYZus{}months} \PY{o}{=} \PY{l+m+mi}{3}
\PY{c+c1}{\PYZsh{} Place the expression in this print statement}
\PY{n}{avg} \PY{o}{=} \PY{n}{summation}\PY{o}{/}\PY{n}{num\PYZus{}months}
\PY{n+nb}{print}\PY{p}{(}\PY{n}{avg}\PY{p}{)}
\end{Verbatim}
\end{tcolorbox}

    \begin{Verbatim}[commandchars=\\\{\}]
39.666666666666664
    \end{Verbatim}

    Quiz: Calculate :

In this quiz you're going to do some calculations for a tiler. Two parts
of a floor need tiling. One part is 9 tiles wide by 7 tiles long, the
other is 5 tiles wide by 7 tiles long. Tiles come in packages of 6.

\begin{enumerate}
\def\labelenumi{\arabic{enumi}.}
\tightlist
\item
  How many tiles are needed?
\item
  You buy 17 packages of tiles containing 6 tiles each. How many tiles
  will be left over?
\end{enumerate}

    \begin{tcolorbox}[breakable, size=fbox, boxrule=1pt, pad at break*=1mm,colback=cellbackground, colframe=cellborder]
\prompt{In}{incolor}{2}{\boxspacing}
\begin{Verbatim}[commandchars=\\\{\}]
\PY{c+c1}{\PYZsh{} Fill this in with an expression that calculates how many tiles are needed.}
\PY{n+nb}{print}\PY{p}{(}\PY{l+m+mi}{9}\PY{o}{*}\PY{l+m+mi}{7} \PY{o}{+} \PY{l+m+mi}{5}\PY{o}{*}\PY{l+m+mi}{7}\PY{p}{)}

\PY{c+c1}{\PYZsh{} Fill this in with an expression that calculates how many tiles will be left over.}
\PY{n+nb}{print}\PY{p}{(}\PY{l+m+mi}{17}\PY{o}{*}\PY{l+m+mi}{6} \PY{o}{\PYZhy{}} \PY{p}{(}\PY{l+m+mi}{9}\PY{o}{*}\PY{l+m+mi}{7} \PY{o}{+} \PY{l+m+mi}{5}\PY{o}{*}\PY{l+m+mi}{7}\PY{p}{)}\PY{p}{)}
\end{Verbatim}
\end{tcolorbox}

    \begin{Verbatim}[commandchars=\\\{\}]
98
4
    \end{Verbatim}

    \hypertarget{variables-and-assignment-operators}{%
\subsubsection{Variables and Assignment
Operators}\label{variables-and-assignment-operators}}

Variables and Assignment Operators: From this page, you will get your
first look at variables in Python. There are three videos in this
concept to show you some different cases you might run into!

    \hypertarget{variables-i}{%
\subsubsection{Variables I}\label{variables-i}}

Variables are used all the time in Python! Below is the example you saw
in the video where we performed the following:

mv\_population = 74728

Here mv\_population is a variable, which holds the value of 74728. This
assigns the item on the right to the name on the left. which is actually
a little different than mathematical equality, as 74728 does not hold
the value of mv\_population.

In any case, whatever term is on the left side, is now a name for
whatever value is on the right side. Once a value has been assigned to a
variable name, you can access the value from the variable name.

    \hypertarget{variables-ii}{%
\subsubsection{Variables II}\label{variables-ii}}

In this video you saw that the following two are equivalent in terms of
assignment:

x = 3 y = 4 z = 5

and

x, y, z = 3, 4, 5

However, the above isn't a great way to assign variables in most cases,
because our variable names should be descriptive of the values they
hold.

Besides writing variable names that are descriptive, there are a few
things to watch out for when naming variables in Python.

\begin{enumerate}
\def\labelenumi{\arabic{enumi}.}
\item
  Only use ordinary letters, numbers and underscores in your variable
  names. They can't have spaces, and need to start with a letter or
  underscore.
\item
  You can't use Python's reserved words, or ``keywords,'' as variable
  names.There are reserved words in every programming language that have
  important purposes, and you'll learn about some of these throughout
  this course. Creating names that are descriptive of the values often
  will help you avoid using any of these keywords. Here you can see a
  \href{https://docs.python.org/3/reference/lexical_analysis.html\#keywords}{table
  of Python's reserved words}.
\item
  The pythonic way to name variables is to use all lowercase letters and
  underscores to separate words.
\end{enumerate}

YES my\_height = 58 my\_lat = 40 my\_long = 105 NO my height = 58 MYLONG
= 40 MyLat = 105

Though the last two of these would work in python, they are not pythonic
ways to name variables. The way we name variables is called snake case,
because we tend to connect the words with underscores.

    Example:

\begin{enumerate}
\def\labelenumi{\arabic{enumi}.}
\tightlist
\item
  mv\_population = 74728
\item
  mv\_population = 74728 + 4000 - 600
\item
  print(mv\_population) \# 78128
\end{enumerate}

Assignment Operators: Below are the assignment operators from the video.
You can also use *= in a similar way, but this is less common than the
operations shown below. You can find some practice with much of what we
have already covered
\href{https://www.programiz.com/python-programming/operators}{here}.

    

    \hypertarget{quiz-variables-and-assignment-operators}{%
\subsubsection{Quiz: Variables and Assignment
Operators}\label{quiz-variables-and-assignment-operators}}

Quiz: Assign and Modify Variables Now it's your turn to work with
variables. The comments in this quiz (the lines that begin with \#) have
instructions for creating and modifying variables. After each comment
write a line of code that implements the instruction.

Note that this code uses
\href{https://en.wikipedia.org/wiki/Scientific_notation}{scientific
notation} to define large numbers. 4.445e8 is equal to 4.445 \emph{10} *
8 which is equal to 444500000.0.

    \begin{tcolorbox}[breakable, size=fbox, boxrule=1pt, pad at break*=1mm,colback=cellbackground, colframe=cellborder]
\prompt{In}{incolor}{7}{\boxspacing}
\begin{Verbatim}[commandchars=\\\{\}]
\PY{c+c1}{\PYZsh{}\PYZsh{} The current volume of a water reservoir (in cubic metres)}
\PY{n}{reservoir\PYZus{}volume} \PY{o}{=} \PY{l+m+mf}{4.445e8}
\PY{c+c1}{\PYZsh{}\PYZsh{} The amount of rainfall from a storm (in cubic metres)}
\PY{n}{rainfall} \PY{o}{=} \PY{l+m+mf}{5e6}

\PY{c+c1}{\PYZsh{}\PYZsh{} decrease the rainfall variable by 10\PYZpc{} to account for runoff}
\PY{n}{rainfall} \PY{o}{*}\PY{o}{=} \PY{l+m+mf}{.9}

\PY{c+c1}{\PYZsh{}\PYZsh{} add the rainfall variable to the reservoir\PYZus{}volume variable}
\PY{n}{reservoir\PYZus{}volume} \PY{o}{+}\PY{o}{=} \PY{n}{rainfall}

\PY{c+c1}{\PYZsh{}\PYZsh{} increase reservoir\PYZus{}volume by 5\PYZpc{} to account for stormwater that flows}
\PY{c+c1}{\PYZsh{}\PYZsh{} into the reservoir in the days following the storm}
\PY{n}{reservoir\PYZus{}volume} \PY{o}{*}\PY{o}{=} \PY{l+m+mf}{1.05}

\PY{c+c1}{\PYZsh{}\PYZsh{} decrease reservoir\PYZus{}volume by 5\PYZpc{} to account for evaporation}
\PY{n}{reservoir\PYZus{}volume} \PY{o}{*}\PY{o}{=} \PY{l+m+mf}{0.95}

\PY{c+c1}{\PYZsh{}\PYZsh{} subtract 2.5e5 cubic metres from reservoir\PYZus{}volume to account for water}
\PY{c+c1}{\PYZsh{}\PYZsh{} that\PYZsq{}s piped to arid regions.}
\PY{n}{reservoir\PYZus{}volume} \PY{o}{\PYZhy{}}\PY{o}{=} \PY{l+m+mf}{2.5e5} 

\PY{c+c1}{\PYZsh{}\PYZsh{} print the new value of the reservoir\PYZus{}volume variable}
\PY{n+nb}{print}\PY{p}{(}\PY{n}{reservoir\PYZus{}volume}\PY{p}{)}
\end{Verbatim}
\end{tcolorbox}

    \begin{Verbatim}[commandchars=\\\{\}]
447627500.0
    \end{Verbatim}

    \hypertarget{integers-and-floats}{%
\subsection{Integers and Floats}\label{integers-and-floats}}

    \hypertarget{integers-and-floats}{%
\subsubsection{Integers and Floats}\label{integers-and-floats}}

There are two Python data types that could be used for numeric values:
\textgreater\textgreater int - for integer values float - for decimal or
floating point values

You can create a value that follows the data type by using the following
syntax:

\begin{quote}
\begin{quote}
x = int(4.7) \# x is now an integer 4 y = float(4) \# y is now a float
of 4.0
\end{quote}
\end{quote}

You can check the type by using the type function:
\textgreater\textgreater\textgreater{} print(type(x)) int
\textgreater\textgreater\textgreater{} print(type(y)) float

    Because the float, or approximation,for 0.1 is actually slightly more
than 0.1, when we add several of them together we can see the difference
between the mathematically correct answer and the one that Python
creates.

\begin{quote}
\begin{quote}
\begin{quote}
print(.1 + .1 + .1 == .3) False
\end{quote}
\end{quote}
\end{quote}

You can see more on this
\href{https://docs.python.org/3/tutorial/floatingpoint.html}{here}.

    \hypertarget{python-best-practices}{%
\subsubsection{Python Best Practices}\label{python-best-practices}}

For all the best practices, \href{https://peps.python.org/pep-0008/}{see
the PEP8 Guidelines}.

You can use the atom package
\href{https://github.blog/2022-06-08-sunsetting-atom/}{linter-python-pep8}
to use pep8 within your own programming environment in the Atom text
editor, but more on this later. If you aren't familiar with text editors
yet, and you are performing all of your programming in the classroom, no
need to worry about this right now.

Follow these guidelines to make other programmers and future you happy!

\begin{quote}
\begin{quote}
Good print(4 + 5)
\end{quote}
\end{quote}

\begin{quote}
\begin{quote}
Bad print( 4 + 5)
\end{quote}
\end{quote}

You should limit each line of code to 80 characters, though 99 is okay
for certain use cases. You can thank
\href{https://softwareengineering.stackexchange.com/questions/148677/why-is-80-characters-the-standard-limit-for-code-width}{IBM
for this ruling.}

Why are these conventions important? Although how you format the code
doesn't affect how it runs, following standard style guidelines makes
code easier to read and consistent among different developers on a team.

    \hypertarget{booleans-comparison-operators-and-logical-operators}{%
\subsection{Booleans, Comparison Operators, and Logical
Operators}\label{booleans-comparison-operators-and-logical-operators}}

    \hypertarget{examples}{%
\paragraph{Examples}\label{examples}}

\begin{quote}
\begin{quote}
x = 42 \textgreater{} 43 \# False age = 14 is\_teen = age \textgreater{}
12 and age \textless{} 20 print(is\_teen) \# True
\end{quote}
\end{quote}

    \hypertarget{booleans-comparison-operators-and-logical-operators}{%
\subsubsection{Booleans, Comparison Operators, and Logical
Operators}\label{booleans-comparison-operators-and-logical-operators}}

The bool data type holds one of the values True or False, which are
often encoded as 1 or 0, respectively. There are 6 comparison operators
that are common to see in order to obtain a bool value:

    \href{https://www.irishtimes.com/news/science/how-george-boole-s-zeroes-and-ones-changed-the-world-1.2014673}{Here}
is more information on how George Boole changed the world!

    \hypertarget{quiz-booleans-comparison-operators-and-logical-operators}{%
\subsection{Quiz: Booleans, Comparison Operators, and Logical
Operators}\label{quiz-booleans-comparison-operators-and-logical-operators}}

    \hypertarget{quiz-which-is-denser-rio-or-san-francisco}{%
\subsubsection{Quiz: Which is denser, Rio or San
Francisco?}\label{quiz-which-is-denser-rio-or-san-francisco}}

Try comparison operators in this quiz! This code calculates the
population densities of Rio de Janeiro and San Francisco.

Write code to compare these densities. Is the population of San
Francisco more dense than that of Rio de Janeiro? Print True if it is
and False if not.

    \begin{tcolorbox}[breakable, size=fbox, boxrule=1pt, pad at break*=1mm,colback=cellbackground, colframe=cellborder]
\prompt{In}{incolor}{1}{\boxspacing}
\begin{Verbatim}[commandchars=\\\{\}]
\PY{n}{sf\PYZus{}population}\PY{p}{,} \PY{n}{sf\PYZus{}area} \PY{o}{=} \PY{l+m+mi}{864816}\PY{p}{,} \PY{l+m+mf}{231.89}

\PY{n}{rio\PYZus{}population}\PY{p}{,} \PY{n}{rio\PYZus{}area} \PY{o}{=} \PY{l+m+mi}{6453682}\PY{p}{,} \PY{l+m+mf}{486.5}

\PY{n}{san\PYZus{}francisco\PYZus{}pop\PYZus{}density} \PY{o}{=} \PY{n}{sf\PYZus{}population}\PY{o}{/}\PY{n}{sf\PYZus{}area}

\PY{n}{rio\PYZus{}de\PYZus{}janeiro\PYZus{}pop\PYZus{}density} \PY{o}{=} \PY{n}{rio\PYZus{}population}\PY{o}{/}\PY{n}{rio\PYZus{}area}

\PY{c+c1}{\PYZsh{} Write code that prints True if San Francisco is denser than Rio, and False otherwise}
\PY{k}{if}\PY{p}{(}\PY{n}{san\PYZus{}francisco\PYZus{}pop\PYZus{}density} \PY{o}{\PYZgt{}} \PY{n}{rio\PYZus{}de\PYZus{}janeiro\PYZus{}pop\PYZus{}density}\PY{p}{)}\PY{p}{:}
    \PY{n+nb}{print}\PY{p}{(}\PY{k+kc}{True}\PY{p}{)}
    
\PY{k}{elif}\PY{p}{(}\PY{n}{rio\PYZus{}de\PYZus{}janeiro\PYZus{}pop\PYZus{}density} \PY{o}{\PYZgt{}} \PY{n}{san\PYZus{}francisco\PYZus{}pop\PYZus{}density}\PY{p}{)}\PY{p}{:}
    \PY{n+nb}{print}\PY{p}{(}\PY{k+kc}{False}\PY{p}{)}
\end{Verbatim}
\end{tcolorbox}

    \begin{Verbatim}[commandchars=\\\{\}]
False
    \end{Verbatim}

    \hypertarget{strings}{%
\subsection{Strings}\label{strings}}

    \hypertarget{strings}{%
\subsubsection{Strings}\label{strings}}

Strings in Python are shown as the variable type str. You can define a
string with either double quotes '' or single quotes '. If the string
you are creating actually has one of these two values in it, then you
need to be careful to assure your code doesn't give an error.

    \begin{quote}
\begin{quote}
\begin{quote}
my\_string = `this is a string!' my\_string2 = ``this is also a
string!!!''
\end{quote}
\end{quote}
\end{quote}

You can also include a ~in your string to be able to include one of
these quotes:

\begin{quote}
\begin{quote}
\begin{quote}
this\_string = `Simon\textquotesingle s skateboard is in the garage.'
print(this\_string)
\end{quote}
\end{quote}
\end{quote}

Simon's skateboard is in the garage.

    If we don't use this, notice we get the following error:

\begin{quote}
\begin{quote}
this\_string = `Simon's skateboard is in the garage.'
\end{quote}
\end{quote}

'\,'\,' File \textless{}``ipython-input-20-e80562c2a290''\textgreater,
line 1 this\_string = `Simon's skateboard is in the garage.' '\,'\,'
\^{} SyntaxError: invalid syntax

    \hypertarget{the-len-function}{%
\subsubsection{The len() function}\label{the-len-function}}

len() is a built-in Python function that returns the length of an
object, like a string. The length of a string is the number of
characters in the string. This will always be an integer.

There is an example above, but here's another one:

\begin{quote}
\begin{quote}
print(len(``ababa'') / len(``ab''))
\end{quote}
\end{quote}

2.5

You know what the data types are for len(``ababa'') and len(``ab'').
Notice the data type of their resulting quotient here.

    \hypertarget{concatenation}{%
\subsubsection{Concatenation}\label{concatenation}}

\begin{quote}
\begin{quote}
\begin{quote}
first\_word = `Hello' second\_word = `There' print(first\_word +
second\_word)
\end{quote}
\end{quote}
\end{quote}

HelloThere

\begin{quote}
\begin{quote}
\begin{quote}
print(first\_word + ' ' + second\_word)
\end{quote}
\end{quote}
\end{quote}

Hello There

\begin{quote}
\begin{quote}
\begin{quote}
print(first\_word * 5)
\end{quote}
\end{quote}
\end{quote}

HelloHelloHelloHelloHello

\begin{quote}
\begin{quote}
\begin{quote}
print(len(first\_word))
\end{quote}
\end{quote}
\end{quote}

5

    \hypertarget{indexing}{%
\subsubsection{Indexing}\label{indexing}}

\begin{quote}
\begin{quote}
\begin{quote}
first\_word{[}0{]}
\end{quote}
\end{quote}
\end{quote}

H

\begin{quote}
\begin{quote}
\begin{quote}
first\_word{[}1{]}
\end{quote}
\end{quote}
\end{quote}

e

    \hypertarget{quiz-strings}{%
\subsection{Quiz: Strings}\label{quiz-strings}}

    \hypertarget{quiz-fix-the-quote}{%
\subsubsection{Quiz: Fix the Quote}\label{quiz-fix-the-quote}}

The line of code in the following quiz will cause a SyntaxError, thanks
to the misuse of quotation marks. First run it with Test Run to view the
error message. Then resolve the problem so that the quote (from
\href{https://www.goodreads.com/author/quotes/203714.Henry_Ford}{Henry
Ford}) is correctly assigned to the variable ford\_quote.

    \begin{tcolorbox}[breakable, size=fbox, boxrule=1pt, pad at break*=1mm,colback=cellbackground, colframe=cellborder]
\prompt{In}{incolor}{2}{\boxspacing}
\begin{Verbatim}[commandchars=\\\{\}]
\PY{c+c1}{\PYZsh{} TODO: Fix this string!}
\PY{n}{ford\PYZus{}quote} \PY{o}{=} \PY{l+s+s1}{\PYZsq{}}\PY{l+s+s1}{Whether you think you can, or you think you can}\PY{l+s+se}{\PYZbs{}\PYZsq{}}\PY{l+s+s1}{t\PYZhy{}\PYZhy{}you}\PY{l+s+se}{\PYZbs{}\PYZsq{}}\PY{l+s+s1}{re right.}\PY{l+s+s1}{\PYZsq{}}

\PY{c+c1}{\PYZsh{} TODO: Fix this string!}
\PY{n}{ford\PYZus{}quote} \PY{o}{=} \PY{l+s+s2}{\PYZdq{}}\PY{l+s+s2}{Whether you think you can, or you think you can}\PY{l+s+s2}{\PYZsq{}}\PY{l+s+s2}{t\PYZhy{}\PYZhy{}you}\PY{l+s+s2}{\PYZsq{}}\PY{l+s+s2}{re right.}\PY{l+s+s2}{\PYZdq{}}
\end{Verbatim}
\end{tcolorbox}

    \hypertarget{quiz-write-a-server-log-message}{%
\subsubsection{Quiz: Write a Server Log
Message}\label{quiz-write-a-server-log-message}}

In this programming quiz, you're going to use what you've learned about
strings to write a logging message for a server.

You'll be provided with example data for a user, the time of their visit
and the site they accessed. You should use the variables provided and
the techniques you've learned to print a log message like this one (with
the username, url, and timestamp replaced with values from the
appropriate variables):

Yogesh accessed the site {[}http://petshop.com/pets/reptiles/pythons{]}
at 16:20.

Use the Test Run button to see your results as you work on coding this
piece by piece.

    \begin{tcolorbox}[breakable, size=fbox, boxrule=1pt, pad at break*=1mm,colback=cellbackground, colframe=cellborder]
\prompt{In}{incolor}{3}{\boxspacing}
\begin{Verbatim}[commandchars=\\\{\}]
\PY{n}{username} \PY{o}{=} \PY{l+s+s2}{\PYZdq{}}\PY{l+s+s2}{Kinari}\PY{l+s+s2}{\PYZdq{}}
\PY{n}{timestamp} \PY{o}{=} \PY{l+s+s2}{\PYZdq{}}\PY{l+s+s2}{04:50}\PY{l+s+s2}{\PYZdq{}}

\PY{n}{url} \PY{o}{=} \PY{l+s+s2}{\PYZdq{}}\PY{l+s+s2}{http://petshop.com/pets/mammals/cats}\PY{l+s+s2}{\PYZdq{}}

\PY{c+c1}{\PYZsh{} TODO: print a log message using the variables above.}

\PY{c+c1}{\PYZsh{} The message should have the same format as this one:}

\PY{c+c1}{\PYZsh{} \PYZdq{}Yogesh accessed the site http://petshop.com/pets/reptiles/pythons at 16:20.\PYZdq{}}


\PY{n+nb}{print}\PY{p}{(}\PY{n}{username}\PY{o}{+}\PY{l+s+s2}{\PYZdq{}}\PY{l+s+s2}{ accessed the site }\PY{l+s+s2}{\PYZdq{}}\PY{o}{+}\PY{n}{url} \PY{o}{+}\PY{l+s+s2}{\PYZdq{}}\PY{l+s+s2}{ at }\PY{l+s+s2}{\PYZdq{}}\PY{o}{+} \PY{n}{timestamp} \PY{o}{+}\PY{l+s+s2}{\PYZdq{}}\PY{l+s+s2}{.}\PY{l+s+s2}{\PYZdq{}}\PY{p}{)}
\end{Verbatim}
\end{tcolorbox}

    \begin{Verbatim}[commandchars=\\\{\}]
Kinari accessed the site http://petshop.com/pets/mammals/cats at 04:50.
    \end{Verbatim}

    \hypertarget{quiz-len}{%
\subsubsection{Quiz: len()}\label{quiz-len}}

Use string concatenation and the len() function to find the length of a
certain movie star's actual full name. Store that length in the
name\_length variable. Don't forget that there are spaces in between the
different parts of a name!

    \begin{tcolorbox}[breakable, size=fbox, boxrule=1pt, pad at break*=1mm,colback=cellbackground, colframe=cellborder]
\prompt{In}{incolor}{4}{\boxspacing}
\begin{Verbatim}[commandchars=\\\{\}]
\PY{n}{given\PYZus{}name} \PY{o}{=} \PY{l+s+s2}{\PYZdq{}}\PY{l+s+s2}{William}\PY{l+s+s2}{\PYZdq{}}
\PY{n}{middle\PYZus{}names} \PY{o}{=} \PY{l+s+s2}{\PYZdq{}}\PY{l+s+s2}{Bradley}\PY{l+s+s2}{\PYZdq{}}
\PY{n}{family\PYZus{}name} \PY{o}{=} \PY{l+s+s2}{\PYZdq{}}\PY{l+s+s2}{Pitt}\PY{l+s+s2}{\PYZdq{}}

\PY{n}{name\PYZus{}length} \PY{o}{=}  \PY{n+nb}{len}\PY{p}{(}\PY{n}{given\PYZus{}name}\PY{o}{+}\PY{l+s+s2}{\PYZdq{}}\PY{l+s+s2}{ }\PY{l+s+s2}{\PYZdq{}}\PY{o}{+}\PY{n}{middle\PYZus{}names}\PY{o}{+}\PY{l+s+s2}{\PYZdq{}}\PY{l+s+s2}{ }\PY{l+s+s2}{\PYZdq{}}\PY{o}{+}\PY{n}{family\PYZus{}name}\PY{p}{)}
\PY{c+c1}{\PYZsh{} todo: calculate how long this name is}
\PY{n}{name\PYZus{}length} \PY{o}{=} \PY{n+nb}{len}\PY{p}{(}\PY{n}{given\PYZus{}name}\PY{p}{)} \PY{o}{+} \PY{n+nb}{len}\PY{p}{(}\PY{n}{middle\PYZus{}names}\PY{p}{)} \PY{o}{+} \PY{n+nb}{len}\PY{p}{(}\PY{n}{family\PYZus{}name}\PY{p}{)} \PY{o}{+} \PY{l+m+mi}{2}

\PY{c+c1}{\PYZsh{} Now we check to make sure that the name fits within the driving license character limit}

\PY{c+c1}{\PYZsh{} Nothing you need to do here}
\PY{n}{driving\PYZus{}license\PYZus{}character\PYZus{}limit} \PY{o}{=} \PY{l+m+mi}{28}

\PY{n+nb}{print}\PY{p}{(}\PY{n}{name\PYZus{}length} \PY{o}{\PYZlt{}}\PY{o}{=} \PY{n}{driving\PYZus{}license\PYZus{}character\PYZus{}limit}\PY{p}{)}
\end{Verbatim}
\end{tcolorbox}

    \begin{Verbatim}[commandchars=\\\{\}]
True
    \end{Verbatim}

    \hypertarget{type-and-type-conversion}{%
\subsection{Type and Type Conversion}\label{type-and-type-conversion}}

    You have seen four data types so far: 1. int 2. float 3. bool 4. string

You got a quick look at type() from an earlier video, and it can be used
to check the data type of any variable you are working with.

\begin{quote}
\begin{quote}
\begin{quote}
print(type(633))
\end{quote}
\end{quote}
\end{quote}

int

\begin{quote}
\begin{quote}
\begin{quote}
print(type(633.0))
\end{quote}
\end{quote}
\end{quote}

float

\begin{quote}
\begin{quote}
\begin{quote}
print(type(`633'))
\end{quote}
\end{quote}
\end{quote}

str

\begin{quote}
\begin{quote}
\begin{quote}
print(type(True))
\end{quote}
\end{quote}
\end{quote}

bool

You saw that you can change variable types to perform different
operations. For example, ``0'' + ``5''

provides completely different output than 0 + 5

What do you think the below would provide? ``0'' + 5

How about the code here: 0 + ``5''

Checking your variable types is really important to assure that you are
retrieving the results you want when programming.

    \hypertarget{quiz-type-and-type-conversion}{%
\subsection{Quiz: Type and Type
Conversion}\label{quiz-type-and-type-conversion}}

    \hypertarget{type-playground}{%
\subsubsection{Type Playground}\label{type-playground}}

Use this programming space with Test Run to experiment with types of
objects. Don't forget to use print to see the output of your code.

    \begin{tcolorbox}[breakable, size=fbox, boxrule=1pt, pad at break*=1mm,colback=cellbackground, colframe=cellborder]
\prompt{In}{incolor}{5}{\boxspacing}
\begin{Verbatim}[commandchars=\\\{\}]
\PY{n+nb}{print}\PY{p}{(}\PY{n+nb}{type}\PY{p}{(}\PY{l+s+s2}{\PYZdq{}}\PY{l+s+s2}{4}\PY{l+s+s2}{\PYZdq{}}\PY{p}{)}\PY{p}{)}
\end{Verbatim}
\end{tcolorbox}

    \begin{Verbatim}[commandchars=\\\{\}]
<class 'str'>
    \end{Verbatim}

    \hypertarget{quiz-total-sales}{%
\subsubsection{Quiz: Total Sales}\label{quiz-total-sales}}

In this quiz, you'll need to change the types of the input and output
data in order to get the result you want.

Calculate and print the total sales for the week from the data provided.
Print out a string of the form ``This week's total sales: xxx'', where
xxx will be the actual total of all the numbers. You'll need to change
the type of the input data in order to calculate that total.

    \begin{tcolorbox}[breakable, size=fbox, boxrule=1pt, pad at break*=1mm,colback=cellbackground, colframe=cellborder]
\prompt{In}{incolor}{9}{\boxspacing}
\begin{Verbatim}[commandchars=\\\{\}]
\PY{n}{mon\PYZus{}sales} \PY{o}{=} \PY{l+s+s2}{\PYZdq{}}\PY{l+s+s2}{121}\PY{l+s+s2}{\PYZdq{}}
\PY{n}{tues\PYZus{}sales} \PY{o}{=} \PY{l+s+s2}{\PYZdq{}}\PY{l+s+s2}{105}\PY{l+s+s2}{\PYZdq{}}
\PY{n}{wed\PYZus{}sales} \PY{o}{=} \PY{l+s+s2}{\PYZdq{}}\PY{l+s+s2}{110}\PY{l+s+s2}{\PYZdq{}}
\PY{n}{thurs\PYZus{}sales} \PY{o}{=} \PY{l+s+s2}{\PYZdq{}}\PY{l+s+s2}{98}\PY{l+s+s2}{\PYZdq{}}
\PY{n}{fri\PYZus{}sales} \PY{o}{=} \PY{l+s+s2}{\PYZdq{}}\PY{l+s+s2}{95}\PY{l+s+s2}{\PYZdq{}}

\PY{c+c1}{\PYZsh{}TODO: Print a string with this format: This week\PYZsq{}s total sales: xxx}

\PY{n}{mon\PYZus{}sales} \PY{o}{=} \PY{n+nb}{int}\PY{p}{(}\PY{n}{mon\PYZus{}sales}\PY{p}{)}
\PY{n}{tues\PYZus{}sales} \PY{o}{=} \PY{n+nb}{int}\PY{p}{(}\PY{n}{tues\PYZus{}sales}\PY{p}{)}
\PY{n}{wed\PYZus{}sales} \PY{o}{=} \PY{n+nb}{int}\PY{p}{(}\PY{n}{wed\PYZus{}sales}\PY{p}{)}
\PY{n}{thurs\PYZus{}sales} \PY{o}{=} \PY{n+nb}{int}\PY{p}{(}\PY{n}{thurs\PYZus{}sales}\PY{p}{)}
\PY{n}{fri\PYZus{}sales} \PY{o}{=} \PY{n+nb}{int}\PY{p}{(}\PY{n}{fri\PYZus{}sales}\PY{p}{)}

\PY{n}{r} \PY{o}{=} \PY{n}{mon\PYZus{}sales} \PY{o}{+} \PY{n}{tues\PYZus{}sales} \PY{o}{+} \PY{n}{wed\PYZus{}sales} \PY{o}{+} \PY{n}{thurs\PYZus{}sales} \PY{o}{+} \PY{n}{fri\PYZus{}sales}

\PY{c+c1}{\PYZsh{}print(r)}
\PY{n+nb}{print}\PY{p}{(}\PY{l+s+s2}{\PYZdq{}}\PY{l+s+s2}{This week}\PY{l+s+s2}{\PYZsq{}}\PY{l+s+s2}{s total sales: }\PY{l+s+s2}{\PYZdq{}}\PY{o}{+}\PY{n+nb}{str}\PY{p}{(}\PY{n}{r}\PY{p}{)}\PY{p}{)}

\PY{c+c1}{\PYZsh{} You will probably need to write some lines of code before the print statement.}
\PY{n}{weekly\PYZus{}sales} \PY{o}{=} \PY{n+nb}{int}\PY{p}{(}\PY{n}{mon\PYZus{}sales}\PY{p}{)} \PY{o}{+} \PY{n+nb}{int}\PY{p}{(}\PY{n}{tues\PYZus{}sales}\PY{p}{)} \PY{o}{+} \PY{n+nb}{int}\PY{p}{(}\PY{n}{wed\PYZus{}sales}\PY{p}{)} \PY{o}{+} \PY{n+nb}{int}\PY{p}{(}\PY{n}{thurs\PYZus{}sales}\PY{p}{)} \PY{o}{+} \PY{n+nb}{int}\PY{p}{(}\PY{n}{fri\PYZus{}sales}\PY{p}{)}
\PY{n}{weekly\PYZus{}sales} \PY{o}{=} \PY{n+nb}{str}\PY{p}{(}\PY{n}{weekly\PYZus{}sales}\PY{p}{)}  \PY{c+c1}{\PYZsh{}convert the type back!!}
\PY{n+nb}{print}\PY{p}{(}\PY{l+s+s2}{\PYZdq{}}\PY{l+s+s2}{This week}\PY{l+s+s2}{\PYZsq{}}\PY{l+s+s2}{s total sales: }\PY{l+s+s2}{\PYZdq{}} \PY{o}{+} \PY{n}{weekly\PYZus{}sales}\PY{p}{)}
\end{Verbatim}
\end{tcolorbox}

    \begin{Verbatim}[commandchars=\\\{\}]
This week's total sales: 529
This week's total sales: 529
    \end{Verbatim}

    \hypertarget{string-methods}{%
\subsection{String Methods}\label{string-methods}}

    In this video you were introduced to methods. Methods are like some of
the functions you have already seen: 1. len(``this'') 2. type(12) 3.
print(``Hello world'')

These three above are functions - notice they use parentheses, and
accept one or more arguments. Functions will be studied in much more
detail in a later lesson!

A method in Python behaves similarly to a function. Methods actually are
functions that are called using dot notation. For example, lower() is a
string method that can be used like this, on a string called ``sample
string'': sample\_string.lower().

Methods are specific to the data type for a particular variable. So
there are some built-in methods that are available for all strings,
different methods that are available for all integers, etc.

    

    You can see that the count and find methods both take another argument.
However, the .islower() method does not accept another argument.

    \begin{tcolorbox}[breakable, size=fbox, boxrule=1pt, pad at break*=1mm,colback=cellbackground, colframe=cellborder]
\prompt{In}{incolor}{1}{\boxspacing}
\begin{Verbatim}[commandchars=\\\{\}]
\PY{n}{my\PYZus{}string} \PY{o}{=} \PY{l+s+s2}{\PYZdq{}}\PY{l+s+s2}{sebastian thrun}\PY{l+s+s2}{\PYZdq{}}
\end{Verbatim}
\end{tcolorbox}

    \begin{tcolorbox}[breakable, size=fbox, boxrule=1pt, pad at break*=1mm,colback=cellbackground, colframe=cellborder]
\prompt{In}{incolor}{2}{\boxspacing}
\begin{Verbatim}[commandchars=\\\{\}]
\PY{n}{my\PYZus{}string}\PY{o}{.}\PY{n}{islower}\PY{p}{(}\PY{p}{)}
\end{Verbatim}
\end{tcolorbox}

            \begin{tcolorbox}[breakable, size=fbox, boxrule=.5pt, pad at break*=1mm, opacityfill=0]
\prompt{Out}{outcolor}{2}{\boxspacing}
\begin{Verbatim}[commandchars=\\\{\}]
True
\end{Verbatim}
\end{tcolorbox}
        
    \begin{tcolorbox}[breakable, size=fbox, boxrule=1pt, pad at break*=1mm,colback=cellbackground, colframe=cellborder]
\prompt{In}{incolor}{3}{\boxspacing}
\begin{Verbatim}[commandchars=\\\{\}]
\PY{n}{my\PYZus{}string}\PY{o}{.}\PY{n}{count}\PY{p}{(}\PY{l+s+s1}{\PYZsq{}}\PY{l+s+s1}{a}\PY{l+s+s1}{\PYZsq{}}\PY{p}{)}
\end{Verbatim}
\end{tcolorbox}

            \begin{tcolorbox}[breakable, size=fbox, boxrule=.5pt, pad at break*=1mm, opacityfill=0]
\prompt{Out}{outcolor}{3}{\boxspacing}
\begin{Verbatim}[commandchars=\\\{\}]
2
\end{Verbatim}
\end{tcolorbox}
        
    \begin{tcolorbox}[breakable, size=fbox, boxrule=1pt, pad at break*=1mm,colback=cellbackground, colframe=cellborder]
\prompt{In}{incolor}{4}{\boxspacing}
\begin{Verbatim}[commandchars=\\\{\}]
\PY{n}{my\PYZus{}string}\PY{o}{.}\PY{n}{find}\PY{p}{(}\PY{l+s+s1}{\PYZsq{}}\PY{l+s+s1}{a}\PY{l+s+s1}{\PYZsq{}}\PY{p}{)}
\end{Verbatim}
\end{tcolorbox}

            \begin{tcolorbox}[breakable, size=fbox, boxrule=.5pt, pad at break*=1mm, opacityfill=0]
\prompt{Out}{outcolor}{4}{\boxspacing}
\begin{Verbatim}[commandchars=\\\{\}]
3
\end{Verbatim}
\end{tcolorbox}
        
    One important string method: format()

    \begin{tcolorbox}[breakable, size=fbox, boxrule=1pt, pad at break*=1mm,colback=cellbackground, colframe=cellborder]
\prompt{In}{incolor}{5}{\boxspacing}
\begin{Verbatim}[commandchars=\\\{\}]
\PY{n}{maria\PYZus{}string} \PY{o}{=} \PY{l+s+s2}{\PYZdq{}}\PY{l+s+s2}{Maria loves }\PY{l+s+si}{\PYZob{}\PYZcb{}}\PY{l+s+s2}{ and }\PY{l+s+si}{\PYZob{}\PYZcb{}}\PY{l+s+s2}{\PYZdq{}} 
\PY{n+nb}{print}\PY{p}{(}\PY{n}{maria\PYZus{}string}\PY{o}{.}\PY{n}{format}\PY{p}{(}\PY{l+s+s2}{\PYZdq{}}\PY{l+s+s2}{math}\PY{l+s+s2}{\PYZdq{}}\PY{p}{,} \PY{l+s+s2}{\PYZdq{}}\PY{l+s+s2}{statistics}\PY{l+s+s2}{\PYZdq{}}\PY{p}{)}\PY{p}{)}
\end{Verbatim}
\end{tcolorbox}

    \begin{Verbatim}[commandchars=\\\{\}]
Maria loves math and statistics
    \end{Verbatim}

    More advanced students can learn more about the formal syntax for using
the format() string method
\href{https://docs.python.org/3.6/library/string.html\#format-string-syntax}{here}.

    \hypertarget{another-important-string-method-split}{%
\subsubsection{Another important string method:
split()}\label{another-important-string-method-split}}

    

    \begin{tcolorbox}[breakable, size=fbox, boxrule=1pt, pad at break*=1mm,colback=cellbackground, colframe=cellborder]
\prompt{In}{incolor}{6}{\boxspacing}
\begin{Verbatim}[commandchars=\\\{\}]
\PY{n}{new\PYZus{}str} \PY{o}{=} \PY{l+s+s2}{\PYZdq{}}\PY{l+s+s2}{The cow jumped over the moon.}\PY{l+s+s2}{\PYZdq{}}
\PY{n}{new\PYZus{}str}\PY{o}{.}\PY{n}{split}\PY{p}{(}\PY{p}{)}
\end{Verbatim}
\end{tcolorbox}

            \begin{tcolorbox}[breakable, size=fbox, boxrule=.5pt, pad at break*=1mm, opacityfill=0]
\prompt{Out}{outcolor}{6}{\boxspacing}
\begin{Verbatim}[commandchars=\\\{\}]
['The', 'cow', 'jumped', 'over', 'the', 'moon.']
\end{Verbatim}
\end{tcolorbox}
        
    \begin{tcolorbox}[breakable, size=fbox, boxrule=1pt, pad at break*=1mm,colback=cellbackground, colframe=cellborder]
\prompt{In}{incolor}{9}{\boxspacing}
\begin{Verbatim}[commandchars=\\\{\}]
\PY{c+c1}{\PYZsh{} Here  the separator is space, and the maxsplit argument is set to 3.}
\PY{n}{new\PYZus{}str} \PY{o}{=} \PY{l+s+s2}{\PYZdq{}}\PY{l+s+s2}{The cow jumped over the moon.}\PY{l+s+s2}{\PYZdq{}}
\PY{n}{new\PYZus{}str}\PY{o}{.}\PY{n}{split}\PY{p}{(}\PY{l+s+s1}{\PYZsq{}}\PY{l+s+s1}{ }\PY{l+s+s1}{\PYZsq{}}\PY{p}{,}\PY{l+m+mi}{3}\PY{p}{)}
\end{Verbatim}
\end{tcolorbox}

            \begin{tcolorbox}[breakable, size=fbox, boxrule=.5pt, pad at break*=1mm, opacityfill=0]
\prompt{Out}{outcolor}{9}{\boxspacing}
\begin{Verbatim}[commandchars=\\\{\}]
['The', 'cow', 'jumped', 'over the moon.']
\end{Verbatim}
\end{tcolorbox}
        
    \begin{tcolorbox}[breakable, size=fbox, boxrule=1pt, pad at break*=1mm,colback=cellbackground, colframe=cellborder]
\prompt{In}{incolor}{10}{\boxspacing}
\begin{Verbatim}[commandchars=\\\{\}]
\PY{c+c1}{\PYZsh{}3. Using \PYZsq{}.\PYZsq{} or period as a separator.}
\PY{n}{new\PYZus{}str} \PY{o}{=} \PY{l+s+s2}{\PYZdq{}}\PY{l+s+s2}{The cow jumped over the moon.}\PY{l+s+s2}{\PYZdq{}}
\PY{n}{new\PYZus{}str}\PY{o}{.}\PY{n}{split}\PY{p}{(}\PY{l+s+s1}{\PYZsq{}}\PY{l+s+s1}{.}\PY{l+s+s1}{\PYZsq{}}\PY{p}{)}
\end{Verbatim}
\end{tcolorbox}

            \begin{tcolorbox}[breakable, size=fbox, boxrule=.5pt, pad at break*=1mm, opacityfill=0]
\prompt{Out}{outcolor}{10}{\boxspacing}
\begin{Verbatim}[commandchars=\\\{\}]
['The cow jumped over the moon', '']
\end{Verbatim}
\end{tcolorbox}
        
    \begin{tcolorbox}[breakable, size=fbox, boxrule=1pt, pad at break*=1mm,colback=cellbackground, colframe=cellborder]
\prompt{In}{incolor}{11}{\boxspacing}
\begin{Verbatim}[commandchars=\\\{\}]
\PY{c+c1}{\PYZsh{}4. Using no separators but having a maxsplit argument of 3.}
\PY{n}{new\PYZus{}str} \PY{o}{=} \PY{l+s+s2}{\PYZdq{}}\PY{l+s+s2}{The cow jumped over the moon.}\PY{l+s+s2}{\PYZdq{}}
\PY{n}{new\PYZus{}str}\PY{o}{.}\PY{n}{split}\PY{p}{(}\PY{k+kc}{None}\PY{p}{,}\PY{l+m+mi}{3}\PY{p}{)}
\end{Verbatim}
\end{tcolorbox}

            \begin{tcolorbox}[breakable, size=fbox, boxrule=.5pt, pad at break*=1mm, opacityfill=0]
\prompt{Out}{outcolor}{11}{\boxspacing}
\begin{Verbatim}[commandchars=\\\{\}]
['The', 'cow', 'jumped', 'over the moon.']
\end{Verbatim}
\end{tcolorbox}
        
    \hypertarget{quiz-string-methods-practice}{%
\subsection{Quiz: String Methods
Practice}\label{quiz-string-methods-practice}}

    \begin{tcolorbox}[breakable, size=fbox, boxrule=1pt, pad at break*=1mm,colback=cellbackground, colframe=cellborder]
\prompt{In}{incolor}{12}{\boxspacing}
\begin{Verbatim}[commandchars=\\\{\}]
\PY{c+c1}{\PYZsh{} Remember, \PYZbs{}n is a special sequence of characters that causes a line break (a new line).}
\PY{n}{verse} \PY{o}{=} \PY{l+s+s2}{\PYZdq{}}\PY{l+s+s2}{If you can keep your head when all about you}\PY{l+s+se}{\PYZbs{}n}\PY{l+s+s2}{  }\PY{l+s+se}{\PYZbs{}}
\PY{l+s+s2}{    Are losing theirs and blaming it on you,}\PY{l+s+se}{\PYZbs{}n}\PY{l+s+s2}{If you can trust yourself when all men doubt you,}\PY{l+s+se}{\PYZbs{}n}\PY{l+s+s2}{ }\PY{l+s+se}{\PYZbs{}}
\PY{l+s+s2}{        But make allowance for their doubting too;}\PY{l+s+se}{\PYZbs{}n}\PY{l+s+s2}{If you can wait and not be tired by waiting,}\PY{l+s+se}{\PYZbs{}n}\PY{l+s+s2}{ }\PY{l+s+se}{\PYZbs{}}
\PY{l+s+s2}{            Or being lied about, don’t deal in lies,}\PY{l+s+se}{\PYZbs{}n}\PY{l+s+s2}{Or being hated, don’t give way to hating,}\PY{l+s+se}{\PYZbs{}n}\PY{l+s+s2}{ }\PY{l+s+se}{\PYZbs{}}
\PY{l+s+s2}{                And yet don’t look too good, nor talk too wise:}\PY{l+s+s2}{\PYZdq{}}
\end{Verbatim}
\end{tcolorbox}

    Use the code editor below to answer the following questions about verse
and use Test Run to check your output in the quiz at the bottom of this
page. 1. What is the length of the string variable verse? 2. What is the
index of the first occurrence of the word `and' in verse? 3. What is the
index of the last occurrence of the word `you' in verse? 4. What is the
count of occurrences of the word `you' in the verse?

You will need to refer to Python's
\href{https://docs.python.org/2/library/string.html}{string methods
documentation}.

    \begin{tcolorbox}[breakable, size=fbox, boxrule=1pt, pad at break*=1mm,colback=cellbackground, colframe=cellborder]
\prompt{In}{incolor}{13}{\boxspacing}
\begin{Verbatim}[commandchars=\\\{\}]
\PY{n}{verse} \PY{o}{=} \PY{l+s+s2}{\PYZdq{}}\PY{l+s+s2}{If you can keep your head when all about you}\PY{l+s+se}{\PYZbs{}n}\PY{l+s+s2}{  Are losing theirs and blaming it on you,}\PY{l+s+se}{\PYZbs{}n}\PY{l+s+s2}{If you can trust yourself when all men doubt you,}\PY{l+s+se}{\PYZbs{}n}\PY{l+s+s2}{  But make allowance for their doubting too;}\PY{l+s+se}{\PYZbs{}n}\PY{l+s+s2}{If you can wait and not be tired by waiting,}\PY{l+s+se}{\PYZbs{}n}\PY{l+s+s2}{  Or being lied about, don’t deal in lies,}\PY{l+s+se}{\PYZbs{}n}\PY{l+s+s2}{Or being hated, don’t give way to hating,}\PY{l+s+se}{\PYZbs{}n}\PY{l+s+s2}{  And yet don’t look too good, nor talk too wise:}\PY{l+s+s2}{\PYZdq{}}
\PY{n+nb}{print}\PY{p}{(}\PY{n}{verse}\PY{p}{)}

\PY{c+c1}{\PYZsh{} Use the appropriate functions and methods to answer the questions above}

\PY{c+c1}{\PYZsh{} Bonus: practice using .format() to output your answers in descriptive messages!}
\PY{n+nb}{print}\PY{p}{(}\PY{l+s+s2}{\PYZdq{}}\PY{l+s+s2}{Verse has a length of }\PY{l+s+si}{\PYZob{}\PYZcb{}}\PY{l+s+s2}{ characters.}\PY{l+s+s2}{\PYZdq{}}\PY{o}{.}\PY{n}{format}\PY{p}{(}\PY{n+nb}{len}\PY{p}{(}\PY{n}{verse}\PY{p}{)}\PY{p}{)}\PY{p}{)}
\PY{n+nb}{print}\PY{p}{(}\PY{l+s+s2}{\PYZdq{}}\PY{l+s+s2}{The first occurence of the word }\PY{l+s+s2}{\PYZsq{}}\PY{l+s+s2}{and}\PY{l+s+s2}{\PYZsq{}}\PY{l+s+s2}{ occurs at the }\PY{l+s+si}{\PYZob{}\PYZcb{}}\PY{l+s+s2}{th index.}\PY{l+s+s2}{\PYZdq{}}\PY{o}{.}\PY{n}{format}\PY{p}{(}\PY{n}{verse}\PY{o}{.}\PY{n}{find}\PY{p}{(}\PY{l+s+s1}{\PYZsq{}}\PY{l+s+s1}{and}\PY{l+s+s1}{\PYZsq{}}\PY{p}{)}\PY{p}{)}\PY{p}{)}
\PY{n+nb}{print}\PY{p}{(}\PY{l+s+s2}{\PYZdq{}}\PY{l+s+s2}{The last occurence of the word }\PY{l+s+s2}{\PYZsq{}}\PY{l+s+s2}{you}\PY{l+s+s2}{\PYZsq{}}\PY{l+s+s2}{ occurs at the }\PY{l+s+si}{\PYZob{}\PYZcb{}}\PY{l+s+s2}{th index.}\PY{l+s+s2}{\PYZdq{}}\PY{o}{.}\PY{n}{format}\PY{p}{(}\PY{n}{verse}\PY{o}{.}\PY{n}{rfind}\PY{p}{(}\PY{l+s+s1}{\PYZsq{}}\PY{l+s+s1}{you}\PY{l+s+s1}{\PYZsq{}}\PY{p}{)}\PY{p}{)}\PY{p}{)}
\PY{n+nb}{print}\PY{p}{(}\PY{l+s+s2}{\PYZdq{}}\PY{l+s+s2}{The word }\PY{l+s+s2}{\PYZsq{}}\PY{l+s+s2}{you}\PY{l+s+s2}{\PYZsq{}}\PY{l+s+s2}{ occurs }\PY{l+s+si}{\PYZob{}\PYZcb{}}\PY{l+s+s2}{ times in the verse.}\PY{l+s+s2}{\PYZdq{}}\PY{o}{.}\PY{n}{format}\PY{p}{(}\PY{n}{verse}\PY{o}{.}\PY{n}{count}\PY{p}{(}\PY{l+s+s1}{\PYZsq{}}\PY{l+s+s1}{you}\PY{l+s+s1}{\PYZsq{}}\PY{p}{)}\PY{p}{)}\PY{p}{)}
\end{Verbatim}
\end{tcolorbox}

    \begin{Verbatim}[commandchars=\\\{\}]
If you can keep your head when all about you
  Are losing theirs and blaming it on you,
If you can trust yourself when all men doubt you,
  But make allowance for their doubting too;
If you can wait and not be tired by waiting,
  Or being lied about, don’t deal in lies,
Or being hated, don’t give way to hating,
  And yet don’t look too good, nor talk too wise:
Verse has a length of 362 characters.
The first occurence of the word 'and' occurs at the 65th index.
The last occurence of the word 'you' occurs at the 186th index.
The word 'you' occurs 8 times in the verse.
    \end{Verbatim}

    \begin{tcolorbox}[breakable, size=fbox, boxrule=1pt, pad at break*=1mm,colback=cellbackground, colframe=cellborder]
\prompt{In}{incolor}{14}{\boxspacing}
\begin{Verbatim}[commandchars=\\\{\}]
\PY{n}{message} \PY{o}{=} \PY{l+s+s2}{\PYZdq{}}\PY{l+s+s2}{Verse has a length of }\PY{l+s+si}{\PYZob{}\PYZcb{}}\PY{l+s+s2}{ characters.}\PY{l+s+se}{\PYZbs{}n}\PY{l+s+s2}{The first occurence of the }\PY{l+s+se}{\PYZbs{}}
\PY{l+s+s2}{word }\PY{l+s+s2}{\PYZsq{}}\PY{l+s+s2}{and}\PY{l+s+s2}{\PYZsq{}}\PY{l+s+s2}{ occurs at the }\PY{l+s+si}{\PYZob{}\PYZcb{}}\PY{l+s+s2}{th index.}\PY{l+s+se}{\PYZbs{}n}\PY{l+s+s2}{The last occurence of the word }\PY{l+s+s2}{\PYZsq{}}\PY{l+s+s2}{you}\PY{l+s+s2}{\PYZsq{}}\PY{l+s+s2}{ }\PY{l+s+se}{\PYZbs{}}
\PY{l+s+s2}{occurs at the }\PY{l+s+si}{\PYZob{}\PYZcb{}}\PY{l+s+s2}{th index.}\PY{l+s+se}{\PYZbs{}n}\PY{l+s+s2}{The word }\PY{l+s+s2}{\PYZsq{}}\PY{l+s+s2}{you}\PY{l+s+s2}{\PYZsq{}}\PY{l+s+s2}{ occurs }\PY{l+s+si}{\PYZob{}\PYZcb{}}\PY{l+s+s2}{ times in the verse.}\PY{l+s+s2}{\PYZdq{}}

\PY{n}{length} \PY{o}{=} \PY{n+nb}{len}\PY{p}{(}\PY{n}{verse}\PY{p}{)}
\PY{n}{first\PYZus{}idx} \PY{o}{=} \PY{n}{verse}\PY{o}{.}\PY{n}{find}\PY{p}{(}\PY{l+s+s1}{\PYZsq{}}\PY{l+s+s1}{and}\PY{l+s+s1}{\PYZsq{}}\PY{p}{)}
\PY{n}{last\PYZus{}idx} \PY{o}{=} \PY{n}{verse}\PY{o}{.}\PY{n}{rfind}\PY{p}{(}\PY{l+s+s1}{\PYZsq{}}\PY{l+s+s1}{you}\PY{l+s+s1}{\PYZsq{}}\PY{p}{)}
\PY{n}{count} \PY{o}{=} \PY{n}{verse}\PY{o}{.}\PY{n}{count}\PY{p}{(}\PY{l+s+s1}{\PYZsq{}}\PY{l+s+s1}{you}\PY{l+s+s1}{\PYZsq{}}\PY{p}{)}

\PY{n+nb}{print}\PY{p}{(}\PY{n}{message}\PY{o}{.}\PY{n}{format}\PY{p}{(}\PY{n}{length}\PY{p}{,} \PY{n}{first\PYZus{}idx}\PY{p}{,} \PY{n}{last\PYZus{}idx}\PY{p}{,} \PY{n}{count}\PY{p}{)}\PY{p}{)}
\end{Verbatim}
\end{tcolorbox}

    \begin{Verbatim}[commandchars=\\\{\}]
Verse has a length of 362 characters.
The first occurence of the word 'and' occurs at the 65th index.
The last occurence of the word 'you' occurs at the 186th index.
The word 'you' occurs 8 times in the verse.
    \end{Verbatim}

    \hypertarget{whats-next}{%
\subsection{What's Next?}\label{whats-next}}

Now that you are familiar with some basic data types and operators, in
the next lesson, you'll learn about data structures, where you organize
and group together these data types into different containers. You'll
also learn about the two remaining types of operators in Python, along
with more useful built-in functions and methods.

    ! jupyter nbconvert testnotebook.ipynb --to python

    jupyter nbconvert --to pdf --output-dir=output\_folder
your\_notebook.ipynb

    \begin{tcolorbox}[breakable, size=fbox, boxrule=1pt, pad at break*=1mm,colback=cellbackground, colframe=cellborder]
\prompt{In}{incolor}{2}{\boxspacing}
\begin{Verbatim}[commandchars=\\\{\}]
\PY{o}{!} jupyter nbconvert \PYZhy{}\PYZhy{}to pdf Data\PYZus{}Types\PYZus{}and\PYZus{}Operators.ipynb
\end{Verbatim}
\end{tcolorbox}

    \begin{Verbatim}[commandchars=\\\{\}]
[NbConvertApp] Converting notebook Data\_Types\_and\_Operators.ipynb to pdf
[NbConvertApp] Writing 64927 bytes to notebook.tex
[NbConvertApp] Building PDF
Traceback (most recent call last):
  File "c:\textbackslash{}users\textbackslash{}monem\textbackslash{}appdata\textbackslash{}local\textbackslash{}programs\textbackslash{}python\textbackslash{}python37\textbackslash{}lib\textbackslash{}runpy.py",
line 193, in \_run\_module\_as\_main
    "\_\_main\_\_", mod\_spec)
  File "c:\textbackslash{}users\textbackslash{}monem\textbackslash{}appdata\textbackslash{}local\textbackslash{}programs\textbackslash{}python\textbackslash{}python37\textbackslash{}lib\textbackslash{}runpy.py",
line 85, in \_run\_code
    exec(code, run\_globals)
  File "C:\textbackslash{}Users\textbackslash{}MONEM\textbackslash{}AppData\textbackslash{}Local\textbackslash{}Programs\textbackslash{}Python\textbackslash{}Python37\textbackslash{}Scripts\textbackslash{}jupyter-
nbconvert.EXE\textbackslash{}\_\_main\_\_.py", line 7, in <module>
  File "c:\textbackslash{}users\textbackslash{}monem\textbackslash{}appdata\textbackslash{}local\textbackslash{}programs\textbackslash{}python\textbackslash{}python37\textbackslash{}lib\textbackslash{}site-
packages\textbackslash{}jupyter\_core\textbackslash{}application.py", line 264, in launch\_instance
    return super(JupyterApp, cls).launch\_instance(argv=argv, **kwargs)
  File "c:\textbackslash{}users\textbackslash{}monem\textbackslash{}appdata\textbackslash{}local\textbackslash{}programs\textbackslash{}python\textbackslash{}python37\textbackslash{}lib\textbackslash{}site-
packages\textbackslash{}traitlets\textbackslash{}config\textbackslash{}application.py", line 846, in launch\_instance
    app.start()
  File "c:\textbackslash{}users\textbackslash{}monem\textbackslash{}appdata\textbackslash{}local\textbackslash{}programs\textbackslash{}python\textbackslash{}python37\textbackslash{}lib\textbackslash{}site-
packages\textbackslash{}nbconvert\textbackslash{}nbconvertapp.py", line 361, in start
    self.convert\_notebooks()
  File "c:\textbackslash{}users\textbackslash{}monem\textbackslash{}appdata\textbackslash{}local\textbackslash{}programs\textbackslash{}python\textbackslash{}python37\textbackslash{}lib\textbackslash{}site-
packages\textbackslash{}nbconvert\textbackslash{}nbconvertapp.py", line 533, in convert\_notebooks
    self.convert\_single\_notebook(notebook\_filename)
  File "c:\textbackslash{}users\textbackslash{}monem\textbackslash{}appdata\textbackslash{}local\textbackslash{}programs\textbackslash{}python\textbackslash{}python37\textbackslash{}lib\textbackslash{}site-
packages\textbackslash{}nbconvert\textbackslash{}nbconvertapp.py", line 498, in convert\_single\_notebook
    output, resources = self.export\_single\_notebook(notebook\_filename,
resources, input\_buffer=input\_buffer)
  File "c:\textbackslash{}users\textbackslash{}monem\textbackslash{}appdata\textbackslash{}local\textbackslash{}programs\textbackslash{}python\textbackslash{}python37\textbackslash{}lib\textbackslash{}site-
packages\textbackslash{}nbconvert\textbackslash{}nbconvertapp.py", line 427, in export\_single\_notebook
    output, resources = self.exporter.from\_filename(notebook\_filename,
resources=resources)
  File "c:\textbackslash{}users\textbackslash{}monem\textbackslash{}appdata\textbackslash{}local\textbackslash{}programs\textbackslash{}python\textbackslash{}python37\textbackslash{}lib\textbackslash{}site-
packages\textbackslash{}nbconvert\textbackslash{}exporters\textbackslash{}exporter.py", line 190, in from\_filename
    return self.from\_file(f, resources=resources, **kw)
  File "c:\textbackslash{}users\textbackslash{}monem\textbackslash{}appdata\textbackslash{}local\textbackslash{}programs\textbackslash{}python\textbackslash{}python37\textbackslash{}lib\textbackslash{}site-
packages\textbackslash{}nbconvert\textbackslash{}exporters\textbackslash{}exporter.py", line 208, in from\_file
    return self.from\_notebook\_node(nbformat.read(file\_stream, as\_version=4),
resources=resources, **kw)
  File "c:\textbackslash{}users\textbackslash{}monem\textbackslash{}appdata\textbackslash{}local\textbackslash{}programs\textbackslash{}python\textbackslash{}python37\textbackslash{}lib\textbackslash{}site-
packages\textbackslash{}nbconvert\textbackslash{}exporters\textbackslash{}pdf.py", line 183, in from\_notebook\_node
    self.run\_latex(tex\_file)
  File "c:\textbackslash{}users\textbackslash{}monem\textbackslash{}appdata\textbackslash{}local\textbackslash{}programs\textbackslash{}python\textbackslash{}python37\textbackslash{}lib\textbackslash{}site-
packages\textbackslash{}nbconvert\textbackslash{}exporters\textbackslash{}pdf.py", line 154, in run\_latex
    self.latex\_count, log\_error, raise\_on\_failure)
  File "c:\textbackslash{}users\textbackslash{}monem\textbackslash{}appdata\textbackslash{}local\textbackslash{}programs\textbackslash{}python\textbackslash{}python37\textbackslash{}lib\textbackslash{}site-
packages\textbackslash{}nbconvert\textbackslash{}exporters\textbackslash{}pdf.py", line 112, in run\_command
    "at \{link\}.".format(formatter=command\_list[0], link=link))
OSError: xelatex not found on PATH, if you have not installed xelatex you may
need to do so. Find further instructions at
https://nbconvert.readthedocs.io/en/latest/install.html\#installing-tex.
    \end{Verbatim}

    \begin{tcolorbox}[breakable, size=fbox, boxrule=1pt, pad at break*=1mm,colback=cellbackground, colframe=cellborder]
\prompt{In}{incolor}{1}{\boxspacing}
\begin{Verbatim}[commandchars=\\\{\}]
\PY{o}{!} jupyter nbconvert \PYZhy{}\PYZhy{}to html Data\PYZus{}Types\PYZus{}and\PYZus{}Operators.ipynb
\end{Verbatim}
\end{tcolorbox}

    \begin{Verbatim}[commandchars=\\\{\}]
[NbConvertApp] Converting notebook Data\_Types\_and\_Operators.ipynb to html
[NbConvertApp] Writing 660804 bytes to Data\_Types\_and\_Operators.html
    \end{Verbatim}

    \begin{tcolorbox}[breakable, size=fbox, boxrule=1pt, pad at break*=1mm,colback=cellbackground, colframe=cellborder]
\prompt{In}{incolor}{ }{\boxspacing}
\begin{Verbatim}[commandchars=\\\{\}]
\PY{o}{!} jupyter nbconvert \PYZhy{}\PYZhy{}to latex Data\PYZus{}Types\PYZus{}and\PYZus{}Operators.ipynb
\end{Verbatim}
\end{tcolorbox}

    .replace(`اللي موجود',`اللي عايوزه', `index')l;


    % Add a bibliography block to the postdoc
    
    
    
\end{document}
